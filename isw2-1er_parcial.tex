\documentclass[a4paper,spanish]{article}

\usepackage[spanish,activeacute]{babel}
\usepackage{moreverb}
\usepackage{fancyhdr}
\usepackage{graphicx}
\usepackage{multicol}

\oddsidemargin 0in
\textwidth 6.2in
\topmargin 0in
\addtolength{\topmargin}{-.5in}
\textheight 10in
\parskip=1ex
\pagestyle{fancy}
%usar el segundo nivel de enumeracion con letras
%\renewcommand{\labelenumii}{\alph{enumii}. }

\newcommand{\rev}[0]{\marginpar{REVISAR}}
\newcommand{\nohecho}[0]{\marginpar{NO HECHO}}

\newcommand{\tab}[0]{\hspace*{0.5cm}}

%noindent en todos lados
\parindent=0in 

\lhead{Ingenier\'ia de Software II}
\rhead{Apunte de repaso primer parcial}

\cfoot{$\thepage$ de \pageref{theend}}


\begin{document}

Disclaimer: Este apunte no es autocontenido y fue pensado como un repaso de los conceptos, 
no para aprenderlos de aqu'i directamente.
\begin{multicols}{2}
\tableofcontents
\end{multicols}

\section{Ingenier\'ia de software}

$\neq$ programaci'on:
\begin{itemize}
\item 1 desarrollador vs equipo
\item sistema simple vs complejo
\item tiempo corto vs largo
\item desarrollador = usuario vs stakeholders
\item poco mantenimiento vs muchisimo
\end{itemize}

$\neq$ ciencias de la computaci'on: parte pr'actica, no te'orica. 
La ISW se nutra de CS.

No silver bullet: 4 dificultades esenciales
\begin{itemize}
\item Complejidad (no lineal con el tama~no)
\item Conformidad (arbitrariedad)
\item Facilidad de cambios
\item Invisibilidad
\end{itemize}

\section{Modelos de ciclo de vida}

Modelos de ciclo de vida: Representaci'on est'andar de etapas de desarrollo,
orden relativo y criterios de transici'on.

Sirve para planificar, elegirlo es una decisi'on cr'itica.

Clave: \textbf{visibilidad}

Plan = modelo + par'ametros (instanciarlo)

$\Rightarrow$ un tradeoff entre rapidez, documentaci'on, calidad, visibilidad,
riesgos, etc

\begin{itemize}
\item Cascada: Requerimientos, luego dise~no, luego implementaci'on 
	(forward only)
\item Cascada con prototipos: Igual, pero se van validando las etapas con
	prototipos. Problema: El prototipo deber'ia tirarse por ser hecho sin
	pensar en calidad y nunca se tira.
\item Sashimi: Cascada con superposici'on de etapas.
\item Cascada con subproyectos: Minicascadas dentro. Soluciona un poco del 
	problema (subproyectos mas cortos $\Rightarrow$ el salto de salm'on es mas 
	corto).
\item Iterativo (hacer varias veces lo mismo) e incremental (producto crece a
	medida que avanza el proyecto).
\begin{itemize}
\item UP, RUP
\item SCRUM
\end{itemize}
\end{itemize}

Requisitos inestables / producto novedoso $\Rightarrow$ iterativo \\
Arquitectura compleja $\Rightarrow$ atacar riesgos desde el inicio \\
Se puede hacer D\&C $\Rightarrow$ iteraciones mas cortas \\
Complejidad en el negocio $\Rightarrow$ cuidar especificaci'on

\section{Planificaci\'on}

Gerenciamiento: Planificaci'on, staffing, control, liderazgo, organizaci'on

Identificar stakeholders: Qui'en usa? Qui'en paga? Qui'en sabe? Qui'en quiere
	que exista? Qui'en es factor de decisi'on para que exista? \\
\tab Claves: sponsor, lider usuario, usuarios directos e indirectos (gente 
	afectada)

Driver: Algo extra'ido de la realidad que dirige el desarrollo. Qu'e tengo
	que intentar optimizar. \\
Restricci'on: Algo que restinge las posibilidades del proyecto. \\
Grado de libertad: Contrario a restricci'on.

Todos se aplican a: Funcionalidad - Calidad - Recursos - Costo - Plazo

Plan += alcance, requerimientos iniciales, QAs

Estimaci'on: Mejora seg'un avanza el proyecto \\
\tab Falla: Optimismo, poca seriedad, no se recuerda bien la experiencia, 
	novedad (falta de experiencia), mala administraci'on de requerimientos
	
\textbf{Estimo | Mido | Registro | Comparo | Analizo | Calibro | Estimo}

>Qu'e estimo? Tama~no, esfuerzo, costo \\
M'etodos de estimaci'on:
\begin{itemize}
\item Algor'itmico: Puntos de funci'on, de objeto, de CU + factores de ajuste
\item Emp'iricos: A ojo, basados en experiencia (proys similares)
\item Wideband delphi (reuni'on convergente)
\end{itemize}

Work Breakdown Structures (WBS): D\&C primero para estimar cosas mas chicas
\begin{itemize}
\item De proceso: Root=proyecto, nodos son tareas
\item De producto: Root=producto, items de software, hardware y datos
\item H'ibrida: Proceso cerca del root, se torna producto llegando a las hojas
\end{itemize}

Dependencias entre tareas (no solo del tipo ``fin a comienzo'') determinan
	dinamicamente las fechas de las tareas. Hitos: Tareas de duraci'on 0. \\
\textbf{<Hay dependencias hacia afuera del proyecto!} con otros procesos de la
empresa que tengan impacto (stakeholders) \\
Dependencias por contenci'on de recursos

Gr'aficos de dependencias
\begin{itemize}
\item Gantt: Barras con fechas, respeta dependencias
\item Pert/CPM: Grafo de dependencias, sirve para buscar camino cr'itico
\end{itemize}

Plan de gesti'on: Documento entregable

L'inea de base: Versi'on freezada del plan, base para el seguimiento

Avance: Peso a hitos entregables ($\neq$ tiempo, $\neq$ esfuerzo). 
	Valor acumulado.
	
\section{Atributos de calidad (QA)}

QA $\simeq$ reqs no funcionales, pero puede que se resuelvan con funcionalidad
 (ej, configuraci'on)

QAs influyen fuertemente la arquitectura, que debe asegurarlos

QAs pueden contraponerse (ej: performance y flexibilidad)

T'acticas de arquitectura: Sirven para asegurar QAs (no son un fin en si 
	mismas)

Especificaci'on: Reqs (funcionales) + QAs + restricciones

Error $\Rightarrow$ defecto $\Rightarrow$ falla (observable por stakeholders)

Primer nivel de taxonom'ia de QAs
\begin{itemize}
\item Disponibilidad: Proba de estar disponible = TiempoMedioHastaFalla /
   (TiempoMedioHastaFalla + TiempoMedioReparaci'on). F'acil de especificar,
   dificil de asegurar.
\item Facilidad de cambios: >Qu'e puede cambiar? Funcionalidad, plataforma,
	otro QA, interfase. >Qui'en lo cambia? Usuario, desarrollador, 
	administrador (configuraci'on, c'odigo, par'ametros)
\item Performance: Latencia, deadline, throughput, jitter, \#reqs no 
	procesados. Dif'icil de expresar.
\item Seguridad: Non-repudiation, confidencialidad, integridad, auditabilidad
\item Facilidad de test: Test harness
\item Usabilidad, escalabilidad, portabilidad
\end{itemize}

QAW (Quality Attributes Workshop): M'etodo del SEI que relaciona a los 
	stakeholders para detectar QAs clave. Brainstorming de escenarios (de 
	ellos se deriva el resto).

\section{Arquitecturas}

\subsection{Introducci\'on}

Arquitectura: Elementos + relaci'on entre ellos + propiedades externamente 
	visibles (interfase / minimo asumible por otros). \\
\tab >Por qu'e? D\&C, plano sistema, comunicaci'on con stakeholders y equipo,
	an'alisis temprano de QA

Arq vs dise~no: Arq solo interfases. In the large.

Diagramas: Orientado a datos, C\&C, cajas y flechas, capas, m'odulos, 
	deployment.
	
Uso de la arquitectura
\begin{itemize}
\item Ingenieros de requerimientos: tradeoff entre reqs (QAs) en 
	competencia
\item Dise~nadores: tradeoff contenci'on de recursos y presupuesto
\item Implementadores: Proveer restricciones y libertades (interfases)
\item Testers/Integradores: Conocer comportamiento caja negra
\item Soporte: an'alisis de impacto preliminar
\item Dise~nadores de otros sistemas: conocer interfase
\item Managers: plano para asignar, planificar y seguimiento
\item Grupo QA: an'alisis de conformidad
\item Gestor de configuraci'on: Organizar repositorios y SCM
\end{itemize}

Vistas: planos de la arquitectura, el sistema no es unidimensional

Clave: detallar vistas relevantes y \textbf{vincularlas}. La relevancia 
	depende del prop'osito. Cada vista expone $\neq$ QAs. \\
\tab \emph{Ninguna vista es \textbf{la} arquitectura}.

Estilos: cliente/servidor, capas, datos compartidos, pipe\&filter, 
	publish\&subscribe \\
\tab Forma de abordar con propiedades conocidas. Se pueden usar $\neq$ en 
	$\neq$ partes del sistema. Hay que documentarlos. Pueden ser desde la 
	forma de encarar la documentaci'on hasta una t'actica.
	
\section{Proceso Unificado (UP)}

UML: Notaci'on est'andar (\textbf{no} es un proceso).

UP: Proceso ``marco'' $\rightarrow$ se adapta a las caracter'isticas 
	particulares del proyecto.
\begin{itemize}
\item Dirigido por CU
\item Centrado en arquitectura (prioritaria siempre, refinamiento progresivo)
\item Iterativo e incremental, los riesgos determinan la construcci'on 
	(administraci'on de requerimientos)
\item Iteraci'on resulta en un sistema (feedback de los usuarios)
\item Time boxing (si no se llega, recortar funcionalidad)
\end{itemize}

Tipos de iteraci'on (fases)
\begin{itemize}
\item Inception: Establece \textbf{caso de negocio}. Especifica. Salidas: Caso
	de negocio, criterios de 'exito, evaluaci'on inicial de riesgos, 
	estimaci'on de recursos y requerimientos (10-20\%).
\item Elaboration: An'alisis, base arq, atacar principales riesgos, plan. 
	Salidas: Modelo de dominio y CU (80\%). Arq testeada y documentada. Caso
	de negocio revisado. Plan de desarrollo.
\item Construction: Especificaci'on, desarrollo y test incremental.
\item Transition: Desarrollo, test e implementaci'on.
\end{itemize}

Hitos: Puntos de control para revisar el avance. Tiene entregables asociados\\
\tab Principales: Fin de fase. Secundarios: Fin de iteraci'on.
 
Disciplinas: Organizan actividades. De desarrollo: Requerimientos, an'alisis,
	arquitectura, dise~no, implementaci'on, test, deploy. De gesti'on: 
	Riesgos, plan, seguimiento, SCM. \\
\tab Generan modelo UML que incluye diagramas.

Artefactos: Informaci'on producida por el equipo (docs, c'odigo, etc).

Workflow: Especifica proceso en t'erminos de actividades, artefactos y 
	workers (con rol).
	
\# artefactos grandes $\Rightarrow$ definir cuales se hacen en un desarrollo
	concreto. \\
\tab artefacto inicial: caso de desarrollo. Decide justamente eso (para cada
	artefacto cu'ando y d'onde se crea y se actualiza).

Ejemplos de artefactos: Modelo CU, modelo de dominio, modelo de an'alisis, 
	modelo de dise~no, de arquitectura, de test, de implementaci'on, 
	prototipos, QA, glosario.
	
\section{Gesti\'on de riesgos}

Riesgo: Problema que todav'ia no ocurri'o 
	$<$probabilidad de que ocurra,impacto$>$. \\
\tab Proba * impacto = exposici'on al riesgo.

\begin{tabular}{ccccc}
\textbf{Identificar} & \textbf{analizar} & \textbf{planificar} & 
	\textbf{seguir} & \textbf{controlar} \\
& & plan contingencia & cuantificar & ejecutar planes
\end{tabular}

Comunicar: Intercambiar info con todos para determinar tempranamente.

Identificar: id $\rightarrow$ doc $\rightarrow$ doc contexto (fuente, 
	relaciones). \\
\tab M'etodos: Brainstorm, reporte peri'odico, cuestionario SEI, reportes
	voluntarios, lista de riesgos comunes. \\
\tab <Al avanzar el proyecto aparecen nuevos riesgos!

Documentaci'on: Dado que [condici'on (fuente del riesgo)] $\Rightarrow$ 
	(posiblemente) [impacto]
	
M'etodo de 3 niveles del SEI para priorizar la lista de riesgos. Probabilidad
	e impacto con 3 niveles, cada combinaci'on tiene una prioridad.

Aproximaci'on: Evitar, reducir probabilidad, reducir impacto (flowchart). 
	Definir alcance y acciones de mitigaci'on. Definir mecanismos de tracking
	(m'etricas que definen si el riesgo est'a ocurriendo).
	
Plan de contingencia: Medir impacto en el plan general. <Tiene que ser 
	implementable!

Riesgos comunes: Producto incrrecto (mal reqs), producto incorrectamente (mal
	QAs), atrasos, costo muy alto.
	
\section{Estimaciones}

<Siempre se puede estimar! aunque con cierto nivel de incertidumbre (baja a
	medida que avanza el proyecto)
\tab Mido a partir de la experiencia (mia o de otros, basado en similitudes
	varias). Es la basa para trabajar.

Problemas: Optimismo, falta de experiencia, omisi'on de tareas (si luego no se
	hacen comprometen la calidad, si se hacen comprometen el cronograma)

>Qu'e estimo? Tama~no, esfuerzo, costo, tiempo (problema: confundirlos!)

M'etodos: Emp'iricos (ojo basado en experiencia), algor'itmicos, 
	descomposici'on (D\&C). Combinaci'on de varios.
	
Clark: $E = \frac{O + 4M + P}{6}$. Elimina un poco sesgo optimista (la 
	medici'on pesimista est'a mas lejos de la media que la optimista)

Wideband delphi: Muchos estimadores. Estimaciones sucesivas revelando 
	progresivamente informaci'on para que converjan. Termina por tiempo o 
	desv'io aceptable.

Puntos de funci'on: Contar entradas/salidas/consultas/archivos l'ogicos/
	interfases. Ponderar por complejidad. Factores de ajuste (complejidades
	grandes)
	
Puntos de objeto (no es POO!): obj = pantallas, reportes y m'odulos 
	(ponderados). Es mas simple y considera el reuso, pero solo sirve para ABM
	
Puntos de CU: Peso CUs y actores ponderadamente. Muchos factores de ajuste.

\section{Seguimiento}

Definici'on de cronogramas: Divisi'on (D\&C), dependencias, asignaci'on de
	tiempo y esfuerzo (fechas cumplen dependencias), responsabilidades, 
	salidas de cada tarea, hitos (chequeo de salidas)

Cronograma efectivo (Tomayko): Detalle de tareas \emph{y recursos (+dif'icil)},
	compatible con planes que interfieran, hitos claros con entregables 
	(permite evaluar completitud - seguimiento).

Seguimiento: Proveer visibilidad a responsables del proyecto para reaccionar
	 (relacion con riesgos). Mirar avance, esfuerzo, calidad (>est'a 
	 \textbf{terminado}?)
	
Avance (valor acumulado): Peso a los hitos, subpeso a las subtareas que hacen
	el hito. Sumo peso de un hito tarea cu'ando: est'a entregado, pas'o SQA,
	fue aprobado por el cliente (puedo ir sumando porcentajes en cada uno de
	esos).
	
Esfuerzo: Detectar desvios $\Rightarrow$ mejorar estimaci'on. Dividir por 
	fase/tarea. Reporte de horas del personal (dif'icil). Es importante para
	el \textbf{tracking de costo}.

Tambi'en seguimiento de: riesgos, QA, proceso definido.

No funciona: Recuperar al final, eliminar QA o test, eliminar requerimientos
	sin acordar con el cliente, desatender integraci'on, mucha 
	sobrededicaci'on. Se requiere asumir el problema de la estimaci'on y 
	corregirlo correctamente.
	
Administraci'on de cambios: Hacerlos controladamente (importante en el SCM) \\
\tab Especialmente administraci'on de requerimientos. \textbf{van} a cambiar,
	estar preparado. Involucrar al usuario, comit'e de aprobaci'on, analizar
	impacto, participan los que hicieron el an'alisis inicial (no olvidar
	cosas anteriores).
	
Management: Liderazgo, delegaci'on.

\section{Mejora de procesos (CMM)}
	
Proceso de desarrollo: \\
\tab IEEE: secuencia de pasos con un prop'osito.
\tab SEI: Conjunto de actividades, m'etodos, pr'acticas y transformaciones
	para dearrollar y mantener software.

Calidad de proceso $\Rightarrow$ calidad de producto

Madurez de un proceso: Nivel de definici'on (calidad). Nivel de 
	administraci'on, control y efectividad.
	
Capacidad de proceso: Rango de resultados a partir de seguirlo.

``Software project performance'': Resultado real. Su varianza (ruido) debe
	controlarse.

Inmadura: Procesos improvisados en cada proyecto, no hay control de procesos,
	m'as dependiente de las personas, menor visibilidad ($\Rightarrow$ menor
	control). \\
Madura: Procesos compartidos por todos, realistas (implementables!), 
	actualizados continuamente, bien definidos, procesos sobreviven a las
	personas.

IDEAL: Initiating - Diagnosing - Establishing - Acting - Learning (forma de
	implementar cambios)

CMMi: Modelo, 5 niveles, cada nivel tiene temas que deben cumplirse para 
	alcanzarlo. Se pueden usar pr'acticas de niveles superiores, pero deben
	tenerse todas para alcanzarlo. \\
\tab Mejoras incrementales: Cada nivel sirve para llegar al otro.

CMMi continuo: Diferentes 'areas, un n'umero de valoraci'on por 'area. 
	M'as flexible.

CMMi: Evoluci'on del CMM (aprendizaje de a~nos), m'as best-practices, mejores
	descripciones, cumple ISO 15504

M'as madurez $\Rightarrow$ m'as visibilidad $\Rightarrow$ m'as capacidad de
	control $\Rightarrow$ menor ruido
	
Cada nivel tiene Key Process Areas (KPAs) que hay que cumplir. Cada KPA tiene
	objetivos y key practices organizadas en 5 secciones (common features).
	
SCAMPI (Standard CMMI Appraisal Method for Process Improvement)
\begin{itemize}
\item Clase A: (full) Foco en institucionalizaci'on. Otorga certificado.
\item Clase B: Foco en deployment. 'Util previo a ciertos procesos.
\item Clase C: Foco en approach. Eval'ua riesgos.
\end{itemize}

Niveles de CMMI
\begin{enumerate}
\item Inicial
\item Managed
\item Defined
\item Quantitatively managed
\item Optimizing
\end{enumerate}

\label{theend}
\end{document}

